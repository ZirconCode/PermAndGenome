\input{header.tex}
\begin{document}

\title{Permutations as Genome Rearrangements}
\author{Simon Gr\"uning}
\address[Simon Gr\"uning]{University of Zurich, R\"{a}mistrasse 71, 8006 Zurich}
\email[Simon Gr\"uning]{\href{mailto:simon.gruening@uzh.ch}{simon.gruening@uzh.ch}}

\maketitle

\clearpage

\section{Block Interchanges}

\begin{definition}
Block Interchange
\end{definition}

\begin{definition}
Block Interchange Distance
$bid(p,q)$
$bid(p) := bid(p,id)$
\end{definition}

\begin{remark}
$bid(p,q)$ is a metric and is left-invariant, just as we have seen for $btd(p,q)$.
\end{remark}

\begin{definition}
Cycle Graph $G(p)$
\end{definition}

\begin{lemma}
The cycle graph $G(p)$ has a unique decomposition into edge-disjoint directed cycles in which the colors of the edges alternate.
\end{lemma}

\begin{proof}
It is easy to see that a path of alternating colours cannot branch since by construction there is always only one edge per color directed away from every node. We begin at the $0$ node and complete a cycle (this must occur since $G(p)$ is finite). Once we have done so, we remove the edges of this cycle from $G(p)$ and iterate. Notice that in removing full cycles the structure is maintained. The cycles are then edge-disjoint by construction.
\end{proof}

\begin{remark}
The cycles are only edge-disjoint not vertex-disjoint.
\end{remark}

\begin{definition}
$c(G(p))$, $c(\Gamma (p))$
From now on the cycles of $G(p)$ will refer to the alternating cycles, not the traditional cycles of a permutation we have seen previously. 
\end{definition}

\begin{example}
p = 123
p = 4213?
p = 4312
draw cycle graph
count cycles
find minimal transpositions to id and show how it works
\end{example}

Before we can find a formula for $bid(p)$ we shall examine how single block interchanges can influence the block interchange distance of a permutation in the following two lemmas.

\begin{lemma}
9.10
Let $p \in S_n$ with $p \neq id$. 
\end{lemma}

\begin{lemma}
9.11

\end{lemma}

\begin{theorem}
9.9
\end{theorem}



\begin{appendix}


\end{appendix}

\end{document}
\input{header.tex}
\begin{document}

\title{Permutations as Genome Rearrangements}
\author{Simon Gr\"uning}
\address[Simon Gr\"uning]{University of Zurich, R\"{a}mistrasse 71, 8006 Zurich}
\email[Simon Gr\"uning]{\href{mailto:simon.gruening@uzh.ch}{simon.gruening@uzh.ch}}

\maketitle

\clearpage



\begin{remark}
RECALL: inversion
\end{remark}


\section{Block Interchanges}

\subsection{The Cycle Graph and Its Relation to $bid$}

\begin{definition}
Block Interchange
\end{definition}

\begin{definition}
Block Interchange Distance
$bid(p,q)$
$bid(p) := bid(p,id)$
\end{definition}

\begin{remark}
$bid(p,q)$ is a metric and is left-invariant, just as we have seen for $btd(p,q)$.
\end{remark}

\begin{definition}
Cycle Graph $G(p)$
\end{definition}

\begin{lemma}
The cycle graph $G(p)$ has a unique decomposition into edge-disjoint directed cycles in which the colors of the edges alternate.
\end{lemma}

\begin{proof}
It is easy to see that a path of alternating colours cannot branch since by construction there is always only one edge per color directed away from every node. We begin at the $0$ node and complete a cycle (this must occur since $G(p)$ is finite). Once we have done so, we remove the edges of this cycle from $G(p)$ and iterate. Notice that in removing full cycles the structure is maintained as we only remove disjointly coloured pairs of edges from every node. The cycles are then edge-disjoint by construction.
\end{proof}

\begin{definition}
$c(G(p))$, $c(\Gamma (p))$
From now on the cycles of $G(p)$ will refer to the alternating cycles, not the traditional cycles of a permutation we have seen previously. 
\end{definition}

\begin{remark}
The cycles are only edge-disjoint not vertex-disjoint.
\end{remark}

\begin{remark}
If there is an alternating path from one node to another, both nodes are in the same cycle.
\end{remark}

\begin{example}
p = 123
p = 4213?
p = 4312
draw cycle graph
count cycles
find minimal transpositions to id and show how it works
\end{example}

\begin{remark}
The identity permutation has $n+1$ cycles. In fact, it is the only permutation with this many cycles.
\end{remark}

Before we can find a formula for $bid(p)$ we shall examine how single block interchanges can influence the block interchange distance of a permutation in the following two lemmas.

\begin{definition}
Canonical Block Interchange of p.
\end{definition}

\begin{lemma}
9.10
Let $p \in S_n$ with $p \neq id$. Then there exists a block interchange which increases $c(G(p))$ by two.
\end{lemma}

\begin{proof}
3 cases
\end{proof}

\begin{lemma}
9.11
A block interchange cannot increase $c(G(p))$ by more than two.
\end{lemma}

\begin{proof}
Proof Sketch Only
\end{proof}

\begin{theorem}
9.9
Let $p \in S_n$. Then $bid(p) = \frac{n+1-c(G(p))}{2}$
\end{theorem}

\begin{proof}
By the previous two lemmata we have for any $p \in S_n$ that
\begin{align*}
\frac{n+1-c(G(p))}{2} \leq bid(p) \leq \frac{n+1-c(G(p))}{2}
\end{align*}
since we require this amount of block interchanges to attain $n+1$ cycles in $p$, thus culminating our desired equality.
\end{proof}

\begin{remark}
We have discovered that we need at most $\left \lfloor{\frac{n}{2}}\right \rfloor $ block interchanges to sort an n-permutation. Furthermore $bid(p) \in \mathbb{N}$ implies that $(n+1)$ and $c(G(p))$ must either both be odd or even.
\end{remark}


\subsection{Average Number of Block Interchanges Required to Sort $p$}

\begin{definition}
Define the Hultman Number as
\begin{align*}
S_H(n,k) := | \{ \pi \in S_n : c(G(p)) = k \} |
\end{align*}
\end{definition}

\begin{example}
$S_H(3,4) = 1$ and $S_H(3,2)=5$.
Draw sample graphs quickly? Maybe at least two.
Continue example after 9.14 et al.
\end{example}


\begin{theorem}
9.14 and Corollary? Proof Sketch?
\end{theorem}

\begin{remark}
The previous theorem illustrates the connection between $c(G(p))$ and $c(\Gamma (p))$. It allows us to translate ?????????? Skip all the other stuff, useless?
\end{remark}

\begin{theorem}
9.20? 9.21, 9.22? just write theorems and quickly talk about at most.
\end{theorem}

\begin{theorem}
9.23. The average number of block interchanges needed to sort an n-permutation is
\begin{align*}
\frac{1}{2} (n- \frac{1}{ \left \lfloor{\frac{n+2}{2}}\right \rfloor} - \sum^n_{i=2} \frac{1}{i} )
\end{align*}
\end{theorem}

\begin{appendix}


\end{appendix}

\end{document}
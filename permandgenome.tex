%%%%%%%%%%%%%%%%%%%%%%%%%%%%%%%%%%%%%%%%%%%%%%%%%%%%%%%%%%%%%%%%%%%%%%%%%%
%Author:																 %
%-------																 %
%Yannis Baehni at University of Zurich									 %
%baehni.yannis@uzh.ch													 %
%																		 %
%Version log:															 %
%------------															 %
%06/02/16 . Basic structure												 %
%04/08/16 . Layout changes including section, contents, abstract.		 %
%05/11/16 . Simon, name changes%
%%%%%%%%%%%%%%%%%%%%%%%%%%%%%%%%%%%%%%%%%%%%%%%%%%%%%%%%%%%%%%%%%%%%%%%%%%

%Page Setup
\documentclass[
	11pt, 
	oneside, 
	a4paper,
	reqno,
	final
]{amsart}

\usepackage{commath}

\usepackage[
	left = 3cm, 
	right = 3cm, 
	top = 3cm, 
	bottom = 3cm
]{geometry}

%Headers and footers
\usepackage{fancyhdr}
	\pagestyle{fancy}
	%Clear fields
	\fancyhf{}
	%Header right
	\fancyhead[R]{
		\footnotesize
		Simon Gr\"uning\\
		\href{mailto:simon.gruening@uzh.ch}{simon.gruening@uzh.ch}
	}
	%Header left
	\fancyhead[L]{
		\footnotesize
		MAT694 Seminar: Introduction to Harmonic Analysis\\
		HS16
	}
	%Page numbering in footer
	\fancyfoot[C]{\thepage}
	%Separation line header and footer
	\renewcommand{\headrulewidth}{0.4pt}
	%\renewcommand{\footrulewidth}{0.4pt}
	
	\setlength{\headheight}{19pt} 

%Title
\usepackage[foot]{amsaddr}
\usepackage{xspace}
\makeatletter
\def\@textbottom{\vskip \z@ \@plus 1pt}
\let\@texttop\relax
\usepackage{etoolbox}
\patchcmd{\abstract}{\scshape\abstractname}{\textbf{\abstractname}}{}{}

%Switching commands for different section formats
%Mainsectionsytle
\newcommand{\mainsectionstyle}{%
  	\renewcommand{\@secnumfont}{\bfseries}
  	\renewcommand\section{\@startsection{section}{1}%
    	\z@{.5\linespacing\@plus.7\linespacing}{-.5em}%
    	{\normalfont\bfseries}}%
	\renewcommand\subsection{\@startsection{subsection}{2}%
    	\z@{.5\linespacing\@plus.7\linespacing}{-.5em}%
    	{\normalfont\bfseries}}%
	\renewcommand\subsubsection{\@startsection{subsubsection}{3}%
    	\z@{.5\linespacing\@plus.7\linespacing}{-.5em}%
    	{\normalfont\bfseries}}%
}
\newcommand{\originalsectionstyle}{%
\def\@secnumfont{\bfseries}%\mdseries
\def\section{\@startsection{section}{1}%
  \z@{.7\linespacing\@plus\linespacing}{.5\linespacing}%
  {\normalfont\bfseries\centering}}
}
%Formatting title of TOC
\renewcommand{\contentsnamefont}{\bfseries}
%Table of Contents
\setcounter{tocdepth}{3}

% Add bold to \section titles in ToC and remove . after numbers
\renewcommand{\tocsection}[3]{%
  \indentlabel{\@ifnotempty{#2}{\bfseries\ignorespaces#1 #2\quad}}\bfseries#3}
% Remove . after numbers in \subsection
\renewcommand{\tocsubsection}[3]{%
  \indentlabel{\@ifnotempty{#2}{\ignorespaces#1 #2\quad}}#3}
\let\tocsubsubsection\tocsubsection% Update for \subsubsection
%...

\newcommand\@dotsep{4.5}
\def\@tocline#1#2#3#4#5#6#7{\relax
  \ifnum #1>\c@tocdepth % then omit
  \else
    \par \addpenalty\@secpenalty\addvspace{#2}%
    \begingroup \hyphenpenalty\@M
    \@ifempty{#4}{%
      \@tempdima\csname r@tocindent\number#1\endcsname\relax
    }{%
      \@tempdima#4\relax
    }%
    \parindent\z@ \leftskip#3\relax \advance\leftskip\@tempdima\relax
    \rightskip\@pnumwidth plus1em \parfillskip-\@pnumwidth
    #5\leavevmode\hskip-\@tempdima{#6}\nobreak
    \leaders\hbox{$\m@th\mkern \@dotsep mu\hbox{.}\mkern \@dotsep mu$}\hfill
    \nobreak
    \hbox to\@pnumwidth{\@tocpagenum{\ifnum#1=1\bfseries\fi#7}}\par% <-- \bfseries for \section page
    \nobreak
    \endgroup
  \fi}
\AtBeginDocument{%
\expandafter\renewcommand\csname r@tocindent0\endcsname{0pt}
}
\def\l@subsection{\@tocline{2}{0pt}{2.5pc}{5pc}{}}
\def\l@subsubsection{\@tocline{2}{0pt}{4.5pc}{5pc}{}}
\makeatother

\advance\footskip0.4cm
\textheight=54pc    %a4paper
\textheight=50.5pc %letterpaper
\advance\textheight-0.4cm
\calclayout

%Font settings
%\usepackage{anyfontsize}
%Footnote settings
%\usepackage{mathptmx}
\usepackage{footmisc}
%	\renewcommand*{\thefootnote}{\fnsymbol{footnote}}

%Further math environments
%Further math fonts (loads amsfonts implicitely)
\usepackage{amssymb}
%Redefinition of \text
%\usepackage{amstext}
\usepackage{upref}
%Graphics
%\usepackage{graphicx}
%\usepackage{caption}
%\usepackage{subcaption}
%Frames
\usepackage{mdframed}
\allowdisplaybreaks
%\usepackage{interval}
\newcommand{\toup}{%
  \mathrel{\nonscript\mkern-1.2mu\mkern1.2mu{\uparrow}}%
}
\newcommand{\todown}{%
  \mathrel{\nonscript\mkern-1.2mu\mkern1.2mu{\downarrow}}%
}
\AtBeginDocument{\renewcommand*\d{\mathop{}\!\mathrm{d}}}
\renewcommand{\Re}{\operatorname{Re}}
\renewcommand{\Im}{\operatorname{Im}}
\DeclareMathOperator\Log{Log}
\DeclareMathOperator\Arg{Arg}
\DeclareMathOperator\sech{sech}

\DeclareMathOperator*\esssup{ess.sup}

%\usepackage{hhline}
%\usepackage{booktabs} 
%\usepackage{array}
%\usepackage{xfrac} 
%\everymath{\displaystyle}
%Enumerate
\usepackage{tikz}
\usetikzlibrary{external}
\tikzexternalize % activate!
\usetikzlibrary{patterns}
\pgfdeclarepatternformonly{adjusted lines}{\pgfqpoint{-1pt}{-1pt}}{\pgfqpoint{40pt}{40pt}}{\pgfqpoint{39pt}{39pt}}%
{
  \pgfsetlinewidth{.8pt}
  \pgfpathmoveto{\pgfqpoint{0pt}{0pt}}
  \pgfpathlineto{\pgfqpoint{39.1pt}{39.1pt}}
  \pgfusepath{stroke}
}
\usepackage{enumitem} 
%\renewcommand{\labelitemi}{$\bullet$}
%\renewcommand{\labelitemii}{$\ast$}
%\renewcommand{\labelitemiii}{$\cdot$}
%\renewcommand{\labelitemiv}{$\circ$}
%Colors
%\usepackage{color}
%\usepackage[cmtip, all]{xy}
%Theorems
\newtheoremstyle{bold}              	 %Name
  {}                                     %Space above
  {}                                     %Space below
  {\itshape}		                     %Body font
  {}                                     %Indent amount
  {\scshape}                             %Theorem head font
  {.}                                    %Punctuation after theorem head
  { }                                    %Space after theorem head, ' ', 
  										 %	or \newline
  {} 
\theoremstyle{bold}
\newtheorem*{definition*}{Definition}
\newtheorem{definition}{Definition}[section]
\newtheorem*{lemma*}{Lemma}
\newtheorem{lemma}{Lemma}[section]
\newtheorem{Proof}{Proof}[section]
\newtheorem{proposition}{Proposition}[section]
\newtheorem{properties}{Properties}[section]
\newtheorem{corollary}{Corollary}[section]
\newtheorem*{theorem*}{Theorem}
\newtheorem{theorem}{Theorem}[section]
\newtheorem{example}{Example}[section]
\newtheorem*{remark*}{Remark}
\newtheorem{remark}{Remark}[section]
%German non-ASCII-Characters
%Graphics-Tool
%\usepackage{tikz}
%\usepackage{tikzscale}
%\usepackage{bbm}
%\usepackage{bera}
%Listing-Setup
%Bibliographie
\usepackage[backend=bibtex, style=alphabetic]{biblatex}
%\usepackage[babel, german = swiss]{csquotes}
\bibliography{Bibliography}
%PDF-Linking
%\usepackage[hyphens]{url}
\usepackage[bookmarksopen=true,bookmarksnumbered=true]{hyperref}
%\PassOptionsToPackage{hyphens}{url}\usepackage{hyperref}
\hypersetup{
  colorlinks   = true, %Colours links instead of ugly boxes
  urlcolor     = blue, %Colour for external hyperlinks
  linkcolor    = blue, %Colour of internal links
  citecolor    = blue %Colour of citations
}
%Weierstrass-P symbol for power set
\newcommand{\powerset}{\raisebox{.15\baselineskip}{\Large\ensuremath{\wp}}}

\usepackage[utf8]{inputenc}
\usepackage[english]{babel}
\usepackage{minted}
\usemintedstyle{pastie}

% Xy-pic for graphs woo
\usepackage{xy}
\usepackage{xypic}
\input xy
\xyoption{all}

\begin{document}

\title{Permutations as Genome Rearrangements}
\author{Simon Gr\"uning}
\address[Simon Gr\"uning]{University of Zurich, R\"{a}mistrasse 71, 8006 Zurich}
\email[Simon Gr\"uning]{\href{mailto:simon.gruening@uzh.ch}{simon.gruening@uzh.ch}}


\maketitle

\section*{(Block Interchanges)}


\clearpage




\section{The Cycle Graph and Block Interchange Distance}

\begin{remark}
Recall an inversion in $p = p_1 p_2 \cdots p_n $ is such that ($ i,j \in [ n ] : p_i > p_j \wedge i < j$).
\end{remark}

\begin{definition}
A \textbf{Block Interchange} is an operation that interchanges two blocks of consecutive entries without rearranging said entries.
\end{definition}

\begin{example}
 $ |34|17|562| \rightarrow |562|17|34| $
\end{example}

\begin{definition}
Let the \textbf{Block Interchange Distance} between two n-permutation $p,q$, denoted as $bid(p,q)$, be the smallest number of block interchanges required to transform $p$ into $q$. Let $bid(p) := bid(p,id)$ be the number of interchanges required to sort $p$.
\end{definition}

\begin{remark}
$bid(p,q)$ is a metric and is left-invariant, just as we have seen for $btd(p,q)$.
\end{remark}

\begin{definition}
Let $p = p_1 p_2 \cdots p_n$ be an n-permutation. The  Cycle Graph $G(p)$ is a graph of coloured directed edges on the vertex set $\{ 0,1, \cdots ,n,n+1 \}$ constructed as follows:
\begin{enumerate}
\item For every $i$ with $1 \leq i \leq n+1$: Add a black edge from $p_i$ to $p_{i-1}$
\item For every $i$ with $0 \leq i \leq n$: Add a red edge from $i$ to $i+1$
\end{enumerate}
with $p_0 = 0$ and $p_{n+1} = n+1$.
\end{definition}

\begin{remark}
The Cycle Graph has $2n+2$ edges for any n-permutation.
\end{remark}

\begin{example} \leavevmode \newline

$p_1 = 123 \implies c(G(p_1))=4$ 

\[  \xygraph{
!{<0cm,0cm>;<1.5cm,0cm>:<0cm,1.5cm>::}
!{(0,0) }*+{\bullet_{0}}="0"
!{(1,0) }*+{\bullet_{1}}="1"
!{(2,0) }*+{\bullet_{2}}="2"
!{(3,0) }*+{\bullet_{3}}="3"
!{(4,0) }*+{\bullet_{4}}="4"
"4":"3"
"3":"2"
"2":"1"
"1":"0"
"0":@[red]@/^0.5cm/"1"
"1":@[red]@/^0.5cm/"2"
"2":@[red]@/^0.5cm/"3"
"3":@[red]@/^0.5cm/"4"
}  \]

$p_2 = 4213 \implies c(G(p_2))=1$ 

\[  \xygraph{
!{<0cm,0cm>;<1.5cm,0cm>:<0cm,1.5cm>::}
!{(0,0) }*+{\bullet_{0}}="0"
!{(1,0) }*+{\bullet_{4}}="4"
!{(2,0) }*+{\bullet_{2}}="2"
!{(3,0) }*+{\bullet_{1}}="1"
!{(4,0) }*+{\bullet_{3}}="3"
!{(5,0) }*+{\bullet_{5}}="5"
"5":"3"
"3":"1"
"1":"2"
"2":"4"
"4":"0"
"0":@[red]@/^0.5cm/"1"
"1":@[red]@/^0.5cm/"2"
"2":@[red]@/^0.5cm/"3"
"3":@[red]@/^1cm/"4"
"4":@[red]@/^1cm/"5"
}  \]

$p_3 = 4312 \implies c(G(p_3))=3$ 

\[  \xygraph{
!{<0cm,0cm>;<1.5cm,0cm>:<0cm,1.5cm>::}
!{(0,0) }*+{\bullet_{0}}="0"
!{(1,0) }*+{\bullet_{4}}="4"
!{(2,0) }*+{\bullet_{3}}="3"
!{(3,0) }*+{\bullet_{1}}="1"
!{(4,0) }*+{\bullet_{2}}="2"
!{(5,0) }*+{\bullet_{5}}="5"
"5":"2"
"2":"1"
"1":"3"
"3":"4"
"4":"0"
"0":@[red]@/^0.5cm/"1"
"1":@[red]@/^0.5cm/"2"
"2":@[red]@/^0.5cm/"3"
"3":@[red]@/^0.5cm/"4"
"4":@[red]@/^1cm/"5"
}  \]

For the final permutation $p_3$ notice that $bid(p_3)=1=\frac{4+1-3}{2} $, something which will become relevant in a later theorem.
\end{example}

\begin{lemma}
The cycle graph $G(p)$ has a unique decomposition into edge-disjoint directed cycles in which the colors of the edges alternate.
\end{lemma}

\begin{proof}
It is easy to see that a path of alternating colours cannot branch since by construction there is always only one edge per color directed away from every node. We begin at the $0$-node and complete a cycle (this must occur since $G(p)$ is finite). Once we have done so, we remove the edges of this cycle from $G(p)$ and iterate. Notice that in removing full cycles the structure is maintained as we only remove disjointly coloured pairs of edges from every node. The cycles are then edge-disjoint by construction.
\end{proof}

\begin{definition}
Let $c(G(p))$ be the \textbf{number of alternating directed cycles} in the decomposition of $G(p)$. To avoid confusion, let $c(\Gamma (p))$ denote the number of cycles in the traditional sense of a permutation. From now on the cycles of $G(p)$ will refer to the alternating cycles.
\end{definition}

\begin{remark} ~ %spacehacks
\begin{enumerate}
\item The cycles are only edge-disjoint not necessarily vertex-disjoint.
\item If there is an alternating path from one node to another, both nodes are in the same cycle.
\item The identity permutation has $n+1$ cycles. In fact, it is the only permutation with this many cycles.
\end{enumerate}
\end{remark}

Before we can find a formula for $bid(p)$ we shall examine how single block interchanges can influence the block interchange distance of a permutation in the following two lemmas.

\begin{definition} Let $p = p_1 p_2 \cdots p_n $ with $ p \neq id $, or in other words, $p$ contains at least one inversion. Select the inversion in $p$ ($ i,j \in [ n ] : p_i > p_j \wedge i < j$) such that $p_j$ is minimal and if necessary, as a second criterion that $p_i$ is maximal. 

Let $x := p_j , y := p_i$ to ease our eyes. By our selection criterion we find that $x-1$ occurs before $y$ in $p$, whilst $y+1$ occurs after $x$. (In the case of $x =1$ simply let $x-1=0$ )
\begin{align*}
p = \cdots (x-1) | \cdots y | \cdots | x \cdots | (y+1) \cdots
\end{align*}

The \textbf{Canonical Block Interchange} is defined as the interchange transforming $p$ into
\begin{align*}
p' = \cdots (x-1) | x \cdots | \cdots | \cdots y | (y+1) \cdots
\end{align*}

\end{definition}

\begin{lemma}
(9.10) Let $p \in S_n$ with $p \neq id$. Then there exists a block interchange which increases $c(G(p))$ by 2.
\end{lemma}

\begin{proof}
We shall prove that the canonical block interchange suffices, or symbolically:
\begin{align*}
 c(G(p')) = c(G(p)) + 2.
\end{align*}

Notice first that any block interchange does not affect the red edges in $G(p)$. The canonical interchange changes 3 black edges when $x$ and $y$ are adjacent, and 4 edges otherwise. We shall perform a proof by cases.

\begin{enumerate}

\item \underline{$x,y $ adjacent:} The cycle decomposition of $G(p)$ will contain an alternating path through the following nodes in respective order:
\begin{align*}
a \to (x-1) \to x \to y \to (y+1) \to b
\end{align*} 
where $a$ is the entry in $p$ directly to the right of $(x-1)$ and $b$ the one to the left of $(y+1)$.
\begin{align*}
p = \cdots (x-1) | a \cdots y | \cdots | x \cdots b | (y+1) \cdots
\end{align*}
Since a path exists, we find that the entries are part of the same cycle $C$. After performing the canonical interchange, $C$ is split into 3 cycles. In the black edges we now have $x \to (x-1)$, $(y+1) \to y$, and $a \to b$ while the red have been untouched as seen in the following illustration:
$$p$$
\[  \xygraph{
!{<0cm,0cm>;<1.5cm,0cm>:<0cm,1.5cm>::}
!{(0,0) }*+{\bullet_{(x-1)}}="(x-1)"
!{(1,0) }*+{\bullet_{a}}="a"
!{(2,0) }*+{\bullet_{y}}="y"
!{(3,0) }*+{\bullet_{x}}="x"
!{(4,0) }*+{\bullet_{b}}="b"
!{(5,0) }*+{\bullet_{(y+1)}}="(y+1)"
"(y+1)":"b"
"x":"y"
"a":"(x-1)"
"y":@[red]@/^0.5cm/"(y+1)"
"(x-1)":@[red]@/^0.5cm/"x"
"b":@[blue]@/^0.5cm/"a"
}  \]

$$\leadsto p'$$

\[  \xygraph{
!{<0cm,0cm>;<1.5cm,0cm>:<0cm,1.5cm>::}
!{(0,0) }*+{\bullet_{(x-1)}}="(x-1)"
!{(1,0) }*+{\bullet_{x}}="x"
!{(2,0) }*+{\bullet_{b}}="b"
!{(3,0) }*+{\bullet_{a}}="a"
!{(4,0) }*+{\bullet_{y}}="y"
!{(5,0) }*+{\bullet_{(y+1)}}="(y+1)"
"(y+1)":"y"
"x":"(x-1)"
"a":"b"
"y":@[red]@/^0.5cm/"(y+1)"
"(x-1)":@[red]@/^0.5cm/"x"
"b":@[blue]@/^0.5cm/"a"
}  \]

Notice the arrow in blue is not necessarily a singular edge, however it is a well defined alternating path.

\item \underline{$x,y $ not adjacent:} We must introduce another notation in this case. Let $a,b,c,d$ be such that they are adjacent to $(x-1), y, x, (y+1)$ respectively pair-wise. Since they are neighbours we will find the alternating paths
\begin{align*}
P_a = a \to (x-1) \to x \to c
\end{align*}
and
\begin{align*}
P_b = b \to y \to y+1 \to d
\end{align*}
We must make a further division of cases at this point.

\begin{enumerate}
\item \underline{$P_a, P_b $ are of one cycle:} Denote the cycle as $C_{ab}$. In this case there exists an alternating path $A_{da}$ from $d$ to $a$ with the final edge being grey. See the following illustration:
$$p$$

\[  \xygraph{
!{<0cm,0cm>;<1.5cm,0cm>:<0cm,1.5cm>::}
!{(0,0) }*+{\bullet_{(x-1)}}="(x-1)"
!{(1,0) }*+{\bullet_{a}}="a"
!{(2,0) }*+{\bullet_{y}}="y"
!{(3,0) }*+{\bullet_{b}}="b"
!{(4,0) }*+{\bullet_{c}}="c"
!{(5,0) }*+{\bullet_{x}}="x"
!{(6,0) }*+{\bullet_{d}}="d"
!{(7,0) }*+{\bullet_{(y+1)}}="(y+1)"
"(y+1)":"d"
"x":"c"
"b":"y"
"a":"(x-1)"
"y":@[red]@/^0.5cm/"(y+1)"
"(x-1)":@[red]@/^0.5cm/"x"
"d":@[blue]@/^0.5cm/"a"
}  \]

$$\leadsto p'$$

\[  \xygraph{
!{<0cm,0cm>;<1.5cm,0cm>:<0cm,1.5cm>::}
!{(0,0) }*+{\bullet_{(x-1)}}="(x-1)"
!{(1,0) }*+{\bullet_{x}}="x"
!{(2,0) }*+{\bullet_{d}}="d"
!{(3,0) }*+{\bullet_{b}}="b"
!{(4,0) }*+{\bullet_{c}}="c"
!{(5,0) }*+{\bullet_{a}}="a"
!{(6,0) }*+{\bullet_{y}}="y"
!{(7,0) }*+{\bullet_{(y+1)}}="(y+1)"
"(y+1)":"y"
"x":"(x-1)"
"b":"d"
"a":"c"
"y":@[red]@/^0.5cm/"(y+1)"
"(x-1)":@[red]@/^0.5cm/"x"
"d":@[blue]@/^0.5cm/"a"
}  \]

where the blue edge is our path $A_{da}$, the red edges are kept static, and once more we have split one cycle into three, all else remaining equal.

\item \underline{$P_a, P_b $ are of distinct cycles:} Denote these cycles $C_a$ and $C_b$ respectively. We find that each cycle shall split into two, we shall cement this only with a short illustration as the argument is analogue.
$$p$$

\[  \xygraph{
!{<0cm,0cm>;<1.5cm,0cm>:<0cm,1.5cm>::}
!{(0,0) }*+{\bullet_{(x-1)}}="(x-1)"
!{(1,0) }*+{\bullet_{a}}="a"
!{(2,0) }*+{\bullet_{y}}="y"
!{(3,0) }*+{\bullet_{b}}="b"
!{(4,0) }*+{\bullet_{c}}="c"
!{(5,0) }*+{\bullet_{x}}="x"
!{(6,0) }*+{\bullet_{d}}="d"
!{(7,0) }*+{\bullet_{(y+1)}}="(y+1)"
"(y+1)":"d"
"x":"c"
"b":"y"
"a":"(x-1)"
"y":@[red]@/^0.5cm/"(y+1)"
"(x-1)":@[red]@/^0.5cm/"x"
"d":@[blue]@/^0.5cm/"b"
"c":@[blue]@/^0.5cm/"a"
}  \]

$$\leadsto p'$$

\[  \xygraph{
!{<0cm,0cm>;<1.5cm,0cm>:<0cm,1.5cm>::}
!{(0,0) }*+{\bullet_{(x-1)}}="(x-1)"
!{(1,0) }*+{\bullet_{x}}="x"
!{(2,0) }*+{\bullet_{d}}="d"
!{(3,0) }*+{\bullet_{b}}="b"
!{(4,0) }*+{\bullet_{c}}="c"
!{(5,0) }*+{\bullet_{a}}="a"
!{(6,0) }*+{\bullet_{y}}="y"
!{(7,0) }*+{\bullet_{(y+1)}}="(y+1)"
"(y+1)":"y"
"x":"(x-1)"
"b":"d"
"a":"c"
"y":@[red]@/^0.5cm/"(y+1)"
"(x-1)":@[red]@/^0.5cm/"x"
"d":@[blue]@/^0.5cm/"b"
"c":@[blue]@/^0.5cm/"a"
}  \]

Where the blue edges are our paths $C_a$ and $C_b$ exempted $(x-1) \to x$ and $y \to (y+1)$ correspondingly.

\end{enumerate}
\end{enumerate}

Having exhausted all cases, we conclude the proof.

\end{proof}

\begin{lemma} (9.11) Let $p \in S_n $. Any block interchange performed on $p$ cannot increase $c(G(p))$ by more than 2.

\end{lemma}


\begin{proof}
We shall only provide a quick sketch of the proof which makes use of the reversibility of block interchanges to create a contradiction.
\begin{enumerate}
\item Any block interchange changes at most 4 black edges of $G(p)$.
\item For a block interchange to create more than 2 cycles, it must break one cycle into 4 smaller cycles.
\item If this where the case, the inverse block interchange would have to turn 4 cycles into one cycle, which leads to contradiction upon closer examination. $\lightning$
\end{enumerate}
\end{proof}


\begin{theorem}
(9.9) Let $p \in S_n$. Then 
\begin{align*}
bid(p) = \frac{n+1-c(G(p))}{2} .
\end{align*}
\end{theorem}

\begin{proof}
By the previous two lemmata we have for any $p \in S_n$ that
\begin{align*}
\frac{n+1-c(G(p))}{2} \leq bid(p) \leq \frac{n+1-c(G(p))}{2}
\end{align*}
since we require a change in $c(G(p))$ of magnitude $( n+1)-c(G(p))$ to achieve 
\begin{align*}
c(G(\tilde{p}))=n+1 \iff \tilde{p} = id
\end{align*}
 and this can be achieved by half the same number of block interchanges at most and at least, thus culminating our desired equality.
\end{proof}

\begin{remark}
We have discovered that we need at most $\left \lfloor{\frac{n}{2}}\right \rfloor $ block interchanges to sort an n-permutation. Furthermore $bid(p) \in \mathbb{N}$ implies that $(n+1)$ and $c(G(p))$ must either both be odd or even.
\end{remark}


\section{Average Number of Block Interchanges Required}


We would now like to find the average number of block interchanges required for any arbitrary n-permutation. We wish to count the cardinality of the set of n-permutations restricted to a specific number of cycles in order to calculate this average.

\begin{definition}
Define the Hultman Number as
\begin{align*}
S_H(n,k) := | \{ \pi \in S_n : c(G(p)) = k \} |
\end{align*}
\end{definition}

\begin{example}
$S_H(3,4) = 1$ and $S_H(3,2)=5$. Thus all 5 non-trivial 3-permutations have 2 cycles.

Have a glance at the graph of $p = 321$:

\[  \xygraph{
!{<0cm,0cm>;<1.5cm,0cm>:<0cm,1.5cm>::}
!{(0,0) }*+{\bullet_{0}}="0"
!{(1,0) }*+{\bullet_{3}}="3"
!{(2,0) }*+{\bullet_{2}}="2"
!{(3,0) }*+{\bullet_{1}}="1"
!{(4,0) }*+{\bullet_{4}}="4"
"4":"1"
"3":"0"
"2":"3"
"1":"2"
"0":@[red]@/^0.5cm/"1"
"1":@[red]@/^0.5cm/"2"
"2":@[red]@/^0.5cm/"3"
"3":@[red]@/^0.5cm/"4"
}  \]

\end{example}

\begin{theorem}
(9.14) 
\begin{align*}
S_H(n,k) = | \{ (q,r) \in S_n^2 : c(\Gamma (r)) = k \wedge qr = (12 \cdots n(n+1)) \} |
\end{align*}
\end{theorem}

The previous theorem illustrates the connection between $c(G(p))$ and $c(\Gamma (p))$. It allows us to translate our question into previously explored territory. After some complex and difficult papers alleged to require things such as symmetric functions, character theory, and non-elementary integration, we find our answer:

\begin{theorem}
(9.23) The average number of block interchanges needed to sort an n-permutation is
\begin{align*}
\frac{1}{2} \left( n- \frac{1}{ \left \lfloor{\frac{n+2}{2}}\right \rfloor} - \sum^n_{i=2} \frac{1}{i} \right)
\end{align*}\end{theorem}

\begin{appendix}


\end{appendix}

\end{document}